\hinttext{!!!ACTION REQUIRED!!!}
\hinttext{The structure I defined is generic and will most likely have to be adapted. I suggest that you skim through the pages and then clear the files \texttt{text/ch2.tex} to \texttt{text/ch7.tex} before you start writing.}

The structure for these kinds of chapters is highly dependent on your research. See \autoref{c:Contribution-2} for a very generic layout.

However, in case you want to use equations you may do so as shown in \autoref{e:Mean} and \autoref{e:Standard-Deviation}.
\begin{eqnarray}
\mu    &=& \frac{1}{n} \sum_{i=1}^{n} x_i
\label{e:Mean} \\
\sigma &=& \sqrt{\frac{1}{n - 1} \sum_{i=1}^{n} (x_i - \mu)^2}
\label{e:Standard-Deviation}
\end{eqnarray}

In case you think that figures are more appropriate, you may use them as demonstrated in \autoref{f:LaTrobe-Logo} and \autoref{f:LaTrobe-Code-of-Arms}. I personally prefer not to place figures explicitly. Instead, I just define \texttt{[t]} to align them at the top of the page and rely on \LaTeX{} to automatically figure out the page on which they should appear. In some cases this results in an extra page being injected. There is no upper page limit for a thesis. However, note that word count limits exist.

\begin{figure}[t]
\centering
\includegraphics[width=0.5\textwidth]{figures/latrobe-logo}
\caption{The logo of La Trobe University.}
\label{f:LaTrobe-Logo}
\end{figure}

\begin{figure}[t]
\centering
\includegraphics[width=4cm]{figures/latrobe-coat-of-arms}
\caption{La Trobe University's coat of arms.}
\label{f:LaTrobe-Code-of-Arms}
\end{figure}

Figures, tables, pseudo-code algorithms and source code listings will automatically be indexed and listed at the beginning of the thesis.


\section{Summary}
\label{s:Contribution-1-Summary}

The final section of each major chapter should summarize the chapter. In comparison to the chapter, the summary should be short ($\frac{1}{2}$ to $2$ pages is normal).